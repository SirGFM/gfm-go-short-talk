\section{What's Go?}

%%================================================================================
%%
\subsection{Brief history}
%%
%%================================================================================

% https://en.wikipedia.org/wiki/Go_(programming_language)
\begin{frame}
    \frametitle{\secname: \small\subsecname\normalsize}

    \begin{itemize}
        \item Designed at Google
        \item Conceived in late 2007
        \item First released in late 2009
        \item Most recent release: 1.17.6
    \end{itemize}

\end{frame}

\begin{frame}
    \frametitle{\secname: \small\subsecname\normalsize}

    \begin{itemize}
        \item Built-in concurrency primitives (goroutines, channels, select)
        \item Composition over inheritance
        \item Garbage collected
        \item Native package system
        \item Easily cross-compiled to many OSes/Architectures
    \end{itemize}

\end{frame}

%%================================================================================
%%
\subsection{Containers in Go}
%%
%%================================================================================

\begin{frame}[fragile]
    \frametitle{\secname: \small\subsecname\normalsize}

    \begin{itemize}
        \item The size of an array is part of its type
        \begin{itemize}
            \item Checked when passed to functions
        \end{itemize}
        \item \texttt{Slice}: Reference to a part of an array
        \begin{itemize}
            \item Has a maximum size (its capacity)
            \item Tracks the current length (number of used slots)
            \item May be further subdivided
        \end{itemize}
        \item Access is always checked!
    \end{itemize}

    \small \begin{lstlisting}[language=c]
var buf [1024]byte
left := buf[:512]
right := buf[512:]
empty_slice := make([]int, 0, 100)
    \end{lstlisting} \normalsize
\end{frame}

\begin{frame}[fragile]
    \frametitle{\secname: \small\subsecname\normalsize}

    \begin{itemize}
        \item \texttt{map}: built-in hash map
        \begin{itemize}
            \item Access must be manually synchronized!
        \end{itemize}
        \item \texttt{range} allows iterating over containers (and channels!!)
    \end{itemize}

    \small \begin{lstlisting}[language=c]
var buf [1024]byte
for idx := range buf { /* ... */ }

for idx, value := range buf { /* ... */ }

m := make(map[string]int)
for key, value := range m {
    if key == "<delete_me>" {
        delete(m, key)
    }
}
    \end{lstlisting} \normalsize
\end{frame}

%%================================================================================
%%
\subsection{Types in Go}
%%
%%================================================================================

% https://go.dev/ref/spec#Properties_of_types_and_values
\begin{frame}[fragile]
    \frametitle{\secname: \small\subsecname\normalsize}

    \begin{itemize}
        \item Renamed types are different types
    \end{itemize}

    % using a slice makes this trickier... which is better, I guess...
    \small \begin{lstlisting}[language=c]
type str_arr1 []string
type str_arr2 []string

func do_stuff(arr *str_arr1) { /* ... */ }

// This would give a compiler error
a2 := str_arr2{"a", "b"}

do_stuff(&a2)
    \end{lstlisting} \normalsize
\end{frame}

\begin{frame}[fragile]
    \frametitle{\secname: \small\subsecname\normalsize}

    \begin{itemize}
        \item Conversion may happen if the underlying type is the same
    \end{itemize}

    \small \begin{lstlisting}[language=c]
type ( str_arr1 []string; str_arr2 []string)

func do_stuff(arr *str_arr1) { /* ... */ }

a2 := str_arr2{"a", "b"}

// This is OK
a1 := str_arr1(a2)
do_stuff(&a1)

// The cast is OK, the call is NOT!
s := []string(a2)
do_stuff(&s)
    \end{lstlisting} \normalsize
\end{frame}

\begin{frame}[fragile]
    \frametitle{\secname: \small\subsecname\normalsize}

    \begin{itemize}
        \item Types may have functions defined for them
    \end{itemize}


    \small \begin{lstlisting}[language=c]
import "math"

type point struct {x float64; y float64}

func (self *point) add(other point) {
    self.x += other.x
    self.y += other.y
}

func (a point) distance(b point) float64 {
    dx := a.x - b.x
    dy := a.y - b.y
    return math.Sqrt(dx*dx + dy*dy)
}
    \end{lstlisting} \normalsize
\end{frame}

\begin{frame}[fragile]
    \frametitle{\secname: \small\subsecname\normalsize}

    \begin{itemize}
        \item Capitalization defines scope
    \end{itemize}


    \small \begin{lstlisting}[language=c]
package some_package

func SomePublicFunction() { /* ... */ }

func somePrivateFunction() { /* ... */ }

type some_struct struct {
    PublicField int
    privateField int
}
    \end{lstlisting} \normalsize
\end{frame}

%%================================================================================
%%
\subsection{Interfaces in Go}
%%
%%================================================================================

\begin{frame}[fragile]
    \frametitle{\secname: \small\subsecname\normalsize}

    \begin{itemize}
        \item Interfaces define generic behaviour
        \item Allows "old code to call new code"
        \item E.g., kNET's certificate signer
    \end{itemize}

    \small \begin{lstlisting}[language=c]
type Signer interface {
    Public() PublicKey

    Sign(rand io.Reader, digest []byte,
            opts SignerOpts) ([]byte, error)
}
    \end{lstlisting} \normalsize
\end{frame}

\begin{frame}[fragile]
    \frametitle{\secname: \small\subsecname\normalsize}

    \small \begin{lstlisting}[language=c]
type Sorter interface {
    Compare(other interface{}) int
}

type Num int

func (a Num) Compare(other interface{}) int {
    b := other.(Num)
    if a > b {
        return 1
    } else if a < b {
        return -1
    }
    return 0
}
    \end{lstlisting} \normalsize
\end{frame}

%%================================================================================
%%
\subsection{Error handling in Go}
%%
%%================================================================================

\begin{frame}
    \frametitle{\secname: \small\subsecname\normalsize}

    \begin{itemize}
        \item Functions may return multiple values (similar to Tuples)
        \item Functions usually return something(s) and an \texttt{error}
        \begin{itemize}
            \item \texttt{nil} indicates success
            \item Otherwise, a value that implements \texttt{Error() string}
        \end{itemize}
        \item \texttt{panic} for fatal errors
        \begin{itemize}
            \item Recoverable, but similar to signals (even segfault)
            \item Sometimes (http package) handled and ignored
        \end{itemize}
    \end{itemize}
\end{frame}

\begin{frame}
    \frametitle{\secname: \small\subsecname\normalsize}

    \begin{itemize}
        \item How to handle errors?
        \begin{itemize}
            \item \texttt{goto}?
            \item Stack unwinding?
            \item Long and messy conditionals?
        \end{itemize}
    \end{itemize}
\end{frame}

\begin{frame}[fragile]
    \frametitle{\secname: \small\subsecname\normalsize}

    \begin{itemize}
        \item \texttt{defer}
        \item Stack of function called when the current function returns
        \item Called even on \texttt{panic}
    \end{itemize}

    \small \begin{lstlisting}[language=c]
f, err := os.Open("some-file")
if err != nil {
    // handle it
}
defer f.Close()
    \end{lstlisting} \normalsize
\end{frame}

%%================================================================================
%%
\subsection{Concurrency in Go}
%%
%%================================================================================

\begin{frame}
    \frametitle{\secname: \small\subsecname\normalsize}

    \begin{itemize}
        \item \texttt{Goroutines}: "lightweight" threads
        \item \texttt{channels}: communication primitive
        \begin{itemize}
            \item Queue/Dequeue items without explicit synchronization
            \item Buffered (queueing blocks until the channel is available)
            \item Unbuffered (gotta go fast)
        \end{itemize}
    \end{itemize}
\end{frame}

\begin{frame}[fragile]
    \frametitle{\secname: \small\subsecname\normalsize}

    \small \begin{lstlisting}[language=c]
c := make(chan interface{}, 1)

go func() {
    for value := range c {
        // do something with value
    }
} ()

// Send values to the goroutine
c <- 0
c <- "aaaa"
// Close the channel to stop the range
close(c)
    \end{lstlisting} \normalsize
\end{frame}

%%================================================================================
%%
\subsection{Standard library}
%%
%%================================================================================

\begin{frame}
    \frametitle{\secname: \small\subsecname\normalsize}

    Great collection of packages in the standard library

    \begin{itemize}
        \item Cryptography
        \item Encoding
        \item Networking
    \end{itemize}
\end{frame}

%%================================================================================
%%
\subsection{Memory management}
%%
%%================================================================================

\begin{frame}[fragile]
    \frametitle{\secname: \small\subsecname\normalsize}

    \begin{itemize}
        \item Go is garbage collected
        \item Automatically allocates objects in the heap as needed
        \item Accessing \texttt{nil} still panics
    \end{itemize}

    \small \begin{lstlisting}[language=c]
func getPointer() *int {
    var i int

    return &i
}
    \end{lstlisting} \normalsize
\end{frame}
