\section{What's GO?}

%%================================================================================
%%
\subsection{Brief history}
%%
%%================================================================================

% https://en.wikipedia.org/wiki/Go_(programming_language)
\begin{frame}
    \frametitle{\secname: \small\subsecname\normalsize}

    \begin{itemize}
        \item Designed at Google
        \item Conceived in late 2007
        \item First released in late 2009
        \item Most recent release: 1.17.6
    \end{itemize}

\end{frame}

\begin{frame}
    \frametitle{\secname: \small\subsecname\normalsize}

    \begin{itemize}
        \item Built-in concurrency primitives (goroutines, channels, select)
        \item Composition over inheritance
        \item Garbage collected
        \item Native package system
        \item Easily cross-compiled to many OSes/Architectures
    \end{itemize}

\end{frame}

%%================================================================================
%%
\subsection{Types in Go}
%%
%%================================================================================

% https://go.dev/ref/spec#Properties_of_types_and_values
\begin{frame}[fragile]
    \frametitle{\secname: \small\subsecname\normalsize}

    \begin{itemize}
        \item Renamed types are different types
    \end{itemize}

    % using a slice makes this trickier... which is better, I guess...
    \small \begin{lstlisting}[language=c]
type str_arr1 []string
type str_arr2 []string

func do_stuff(arr *str_arr1) { /* ... */ }

// This would give a compiler error
a2 := str_arr2{"a", "b"}

do_stuff(&a2)
    \end{lstlisting} \normalsize
\end{frame}

\begin{frame}[fragile]
    \frametitle{\secname: \small\subsecname\normalsize}

    \begin{itemize}
        \item Conversion may happen if the underlying type is the same
    \end{itemize}

    \small \begin{lstlisting}[language=c]
type ( str_arr1 []string; str_arr2 []string)

func do_stuff(arr str_arr1) { /* ... */ }

a2 := str_arr2{"a", "b"}

// This is OK
a1 := str_arr1(a2)
do_stuff(&a1)

// The cast is OK, the call is NOT!
s := []string(a2)
do_stuff(&s)
    \end{lstlisting} \normalsize
\end{frame}

\begin{frame}[fragile]
    \frametitle{\secname: \small\subsecname\normalsize}

    \begin{itemize}
        \item Types may have functions defined for them
    \end{itemize}


    \small \begin{lstlisting}[language=c]
import "math"

type point struct {x float64; y float64}

func (self *point) add(other point) {
    self.x += other.x
    self.y += other.y
}

func (a point) distance(b point) float64 {
    dx := a.x - b.x
    dy := a.y - b.y
    return math.Sqrt(dx*dx + dy*dy)
}
    \end{lstlisting} \normalsize
\end{frame}

\begin{frame}[fragile]
    \frametitle{\secname: \small\subsecname\normalsize}

    \begin{itemize}
        \item Capitalization defines scope
    \end{itemize}


    \small \begin{lstlisting}[language=c]
package some_package

func SomePublicFunction() { /* ... */ }

func somePrivateFunction() { /* ... */ }

type some_struct struct {
    PublicField int
    privateField int
}
    \end{lstlisting} \normalsize
\end{frame}

%%================================================================================
%%
\subsection{Interfaces in Go}
%%
%%================================================================================

\begin{frame}[fragile]
    \frametitle{\secname: \small\subsecname\normalsize}

    \begin{itemize}
        \item Interfaces define generic behaviour
        \item Allows "old code to call new code"
        \item E.g., kNET's certificate signer
    \end{itemize}

    \small \begin{lstlisting}[language=c]
type Signer interface {
    Public() PublicKey

    Sign(rand io.Reader, digest []byte,
            opts SignerOpts) ([]byte, error)
}
    \end{lstlisting} \normalsize
\end{frame}

\begin{frame}[fragile]
    \frametitle{\secname: \small\subsecname\normalsize}

    \small \begin{lstlisting}[language=c]
type Sorter interface {
    Compare(other interface{}) int
}

type Num int

func (a Num) Compare(other interface{}) int {
    b := other.(Num)
    if a > b {
        return 1
    } else if a < b {
        return -1
    }
    return 0
}
    \end{lstlisting} \normalsize
\end{frame}
