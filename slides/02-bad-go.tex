\section{Go...tchas}

%%================================================================================
%%
\subsection{nil interface vs nil value}
%%
%%================================================================================

\begin{frame}
    \frametitle{\secname: \small\subsecname\normalsize}

    An interface is always nilable, but it may also contain a nilable type. \break

    Example: https://go.dev/play/p/owF-aeX5G7C \break

    Alternative: https://go.dev/play/p/awn8gHExXbq
\end{frame}

%%================================================================================
%%
\subsection{Defer without checking errors}
%%
%%================================================================================

\begin{frame}
    \frametitle{\secname: \small\subsecname\normalsize}

    Although deferred simplifies releasing resources on error, it can lead to some tricky errors. \break

    Example: https://go.dev/play/p/obWN8ih6fmL
\end{frame}

%%================================================================================
%%
\subsection{Shadowing}
%%
%%================================================================================

\begin{frame}
    \frametitle{\secname: \small\subsecname\normalsize}

    cgo (Google's compiler) does not generate warnings for shadowed variables. \break

    There's a tool to detect that. \break

    Example: https://go.dev/play/p/h7r4n9PVQvz
\end{frame}

%%================================================================================
%%
\subsection{Wrapping nil errors}
%%
%%================================================================================

\begin{frame}
    \frametitle{\secname: \small\subsecname\normalsize}

    Being able to wrap errors can be quite a powerful tool. \break

    github.com/pkg/errors allows wrapping the error and its callstack. \break

    However, if a function indicates an error on something other than an \texttt{error}, this can lead to issues... \break

    Example: https://go.dev/play/p/61rfe7ojw8o
\end{frame}

%%================================================================================
%%
\subsection{Writing past a slice}
%%
%%================================================================================

\begin{frame}
    \frametitle{\secname: \small\subsecname\normalsize}

    By writing to a slice that's a section of a larger slice/array, you may end-up overwriting the original container. \break

    Example: https://go.dev/play/p/CG-fBKey0Yr
\end{frame}
